\documentclass[11pt, oneside]{article}   	% use "amsart" instead of "article" for AMSLaTeX format
%\usepackage{geometry}                		% See geometry.pdf to learn the layout options. There are lots.
%\geometry{letterpaper}                   		% ... or a4paper or a5paper or ... 
%\geometry{landscape}                		% Activate for rotated page geometry
%\usepackage[parfill]{parskip}    		% Activate to begin paragraphs with an empty line rather than an indent
\usepackage{graphicx}				% Use pdf, png, jpg, or eps§ with pdflatex; use eps in DVI mode
								% TeX will automatically convert eps --> pdf in pdflatex		
\usepackage{amsmath,amssymb}

%SetFonts

%SetFonts


\title{Bernoulli-Euler beam theory}
\author{Peter Mackenzie-Helnwein}
%\date{}							% Activate to display a given date or no date

\begin{document}
\maketitle

\section{Introduction}
This tool employs the Bernoulli-Euler beam theory.  This theory, also known as \emph{shear rigid beam theory}, is based on the kinematic assumption  that
\begin{quote}
   Any plane cross section perpendicular to the undeformed beam's axis remain plane and perpendicular to the axis throughout the deformation.
\end{quote}
This allows us to reduce the three-dimensional problem to a single unknown function, $v(x)$, known as \emph{deflection} of the beam.

\section{Kinematics}
Navier's assumption leads to
\begin{equation}
	u(x,y) = -y \,v'(x)
	\qquad
	v(x,y) = v(x)
	\label{Eq:1}
\end{equation}
\begin{equation}
	\varepsilon(x,y) = \frac{\partial u(x,y)}{\partial x} = -y\, v''(x)
	\label{Eq:2}
\end{equation}

\section{Constitutive relations}
\begin{equation}
	\sigma(x,y) = E\,\varepsilon(x,y)
	\label{Eq:3}
\end{equation}
\begin{equation}
	\sigma(x,y) = -E y\,v''(x)
	\label{Eq:4}
\end{equation}

\section{Stress resultants}
The beam sees two stress resultants: the internal moment, 
\begin{equation}
	M(x) = -\int_A y\sigma(x,y)\, dA
	\label{Eq:5}
\end{equation}
and the transverse shear force,
\begin{equation}
	V(x)  = -\int_A \tau_{xy}(x,y)\, dA
	\label{Eq:6}
\end{equation}
Substituting \eqref{Eq:4} into \eqref{Eq:5} yields
\begin{equation}
	M(x) = \int_A E y^2 \, v''(x) \, dA = E\,v''(x) \int_A y^2\,dA = EI\,v''(x)
	\label{Eq:7}
\end{equation}
where 
\begin{equation}
	I = \int_A y^2\,dA
	\label{Eq:7b}
\end{equation}
is the \emph{area moment of inertia} or, short, \emph{moment of inertia}.

Note that the modulus of elasticity, $E$, characterizes the material, the moment of inertia, $I$, characterized the shape of the cross section, and the second derivative of the deflection, $v''(x)$, characterizes the deformation (curvature) of the beam.

\section{Equilibrium}
Equilibrium is formulated in terms of shear forces, $V(x)$, and internal moments, $M(x)$.
Equilibrium of forces on an beam element of infinitesimal length, formulated in the $y$-direction, yields
\begin{equation}
	V'(x) = -w(x)
	\label{Eq:8}
\end{equation}
where $w(x)$ is the distributed lateral load per length.  $w(x)$ is defined positive if pointing against the (upward) positive $y$-axis.

Moment equilibrium around the out-of-plane axis on the same element yields
\begin{equation}
	M'(x) = V(x) ~.
	\label{Eq:9}
\end{equation}
A system for which equations~\eqref{Eq:8} and \eqref{Eq:9} are sufficient to determine the internal moment and shear functions is called \emph{statically determinate}. Otherwise, the system is called \emph{statically indeterminate}. Solving these equations for the latter requires consideration of the kinematic relation~\eqref{Eq:7} and respective boundary conditions.

Equations~\eqref{Eq:8} and \eqref{Eq:9} may be combined into one equation as
\begin{equation}
	M''(x) = V'(x)=-w(x)
	\label{Eq:10}
\end{equation}
Equation~\eqref{Eq:10} replaces both equilibrium equations \eqref{Eq:8} and \eqref{Eq:9}.

\section{Governing equation}
The governing equation is obtained by assuming the displacement function, $v(x)$, as the primary unknown and expressing $M(x)$ in \eqref{Eq:10} using \eqref{Eq:7} to obtain
\begin{equation}
	\left( EI(x)\,v''(x) \right)'' + w(x) = 0
	\label{Eq:11}
\end{equation}
This equation is known as the governing equation of the Bernoulli-Euler beam.

If the beam possesses a constant cross section and is made of one material, then $EI(x)=EI=const.$ and
\eqref{Eq:11} simplifies to
\begin{equation}
	EI\,v''''(x) + w(x) = 0
	\label{Eq:12}
\end{equation}
Equation~\eqref{Eq:12} is what is implemented in this program.

\section{Finding moment, shear force, and slope from the displacement function}
Solving \eqref{Eq:12} and applying suitable boundary conditions yields the displacement function, $v(x)$, for the beam.
The slope, $\theta(x)$, is obtained through differentiation as
\begin{equation}
	\theta(x) = v'(x) ~.
	\label{Eq:13}
\end{equation}
It is positive if the cross section rotates counter-clockwise during deformation.

The moment follows from \eqref{Eq:7} as
\begin{equation}
	M(x) = EI(x)\, v''(x) = EI(x)\, \theta'(x) ~.
	\label{Eq:14}
\end{equation}
The transverse shear force follows from~\eqref{Eq:10} as
\begin{equation}
	V(x) = M'(x) = \left( EI(x)\, v''(x) \right)'
	\label{Eq:15}
\end{equation}
or, for constant $EI$, simplifies to
\begin{equation}
	V(x) = EI\, v'''(x) ~.
	\label{Eq:16}
\end{equation}
\end{document}  